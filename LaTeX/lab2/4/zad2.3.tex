\documentclass[a4paper,12pt]{article}
\usepackage[T1]{fontenc}
\usepackage[latin2]{inputenc}
\usepackage[polish]{babel}
\usepackage{amssymb}
\usepackage{amsthm}
\usepackage{times}
\usepackage{anysize}
\marginsize{1.4cm}{1.4cm}{1.5cm}{1.5cm}
\sloppy 
\theoremstyle{definition}
\newtheorem{df}{Definicja}
\begin{document}
	Istnieje �cis�y zwi�zek mi�dzy rozk�adem macierzy A na macierze L i U a metod� eliminacji Gaussa. Mo�na wykaza�, �e elementy kolejnych kolumn macierzy L s� r�wne wsp�czynnikom przez kt�re mno�one s� w kolejnych krokach wiersze uk�adu r�wna� celem dokonania eliminacji niewiadomych w odpowiednich kolumnach. Natomiast macierz U jest r�wna macierzy tr�jk�tnej uzyskanej w eliminacji Gaussa.
	$$\mathbf{[A|b]} = \left [ \begin{array}{rrrr}
	2&2&4&4\\
	1&2&2&4\\
	1&4&1&1\\
	\end{array} \right]
	= \left [ \begin{array}{rrrr}
	2&2&4&4\\
	0&1&0&2\\
	0&3&-1&-1\\
	\end{array} \right]
	= \left [ \begin{array}{rrrr}
	2&2&4&4\\
	0&1&0&2\\
	0&0&-1&-7\\
	\end{array} \right]
	$$
	$$\mathbf{L} = \left [ \begin{array}{rrr}
	1&0&0\\
	\frac{1}{2}&1&0\\
	\frac{1}{2}&3&1\\
	\end{array} \right]
	\mathbf{U} = \left [ \begin{array}{rrrr}
		2&2&4&4\\
		0&1&0&2\\
		0&0&-1&-7\\
	\end{array} \right]
	$$
\end{document}