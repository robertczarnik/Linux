\documentclass[a4paper,4pt]{article}      
\usepackage[utf8]{inputenc} 
\usepackage[T1]{fontenc}
\usepackage{times}
\usepackage{amssymb}
\usepackage{amsthm}

\usepackage[polish]{babel}
\newtheorem{tw}{\hspace{12mm}\em{Przykład}}[section]
\usepackage{times}
\usepackage{latexsym}
\sloppy

\begin{document}
\setcounter{section}{8}
\setcounter{tw}{2}
\begin{tw}  Wykażemy, że funkcje: \end{tw}
$$\centering  \begin {array}{c} 
$ f(x)=-arc tg x$ \hspace{7mm} i \hspace{7mm}  $g(x)=arc cos $\frac{x}{\sqrt{1+x^2}}}$ $ \end{array} $$
różnią się jedynie o stałą $B=-\frac{\pi}{2}$.\\

Dla każdego x \in \textbf{R}, mamy:\\ 

$$\centering  \begin {array}{c} 
$f'(x)=$\frac{-1}{1+x^2}\\ \\
$g'(x)=$\frac{-1}{\sqrt{1-\left(\frac{x}{\sqrt{1+x^2}}\right )^2}}$\cdot \frac{\sqrt{1+x^2}-\frac{2x^2}{2\sqrt{1+x^2}}}{1+x^2}=\frac{-1}{1+x^2}$; $$ 
\end{array} $$
Oznacza to, że:\\
$$\centering  \begin {array}{c} 
$f'(x)=g'(x)$ \end{array} $$
więc na podstawie ostatniego wniosku możemy napisać:\\
$$\centering  \begin {array}{c} 
$\forall_x \in \textbf{R}: f(x)=g(x)+B. \end{array} $$
Jednocześnie, np. dla x=0$, mamy:\\

$$\centering f(0)=0, g(0)=\frac{\pi}{2},$$
%$f(0)=0, g(0)=\frac{\pi}{2},$\\
zatem nietrudno zauważyć, że ostatnia równość ma miejsce, gdy $B=-\frac{\pi}{2}$

\right) 
$



\end{document}