\documentclass[a4paper,12pt,twoside]{article}
\usepackage[utf8]{inputenc}
\usepackage[T1]{fontenc}
\usepackage{graphicx}
\usepackage{anysize}
\usepackage{enumerate}
\usepackage{amssymb}
\usepackage[polish]{babel}


%\marginsize{left}{right}{top}{bottom}
\marginsize{2.5cm}{2.5cm}{2.5cm}{2.5cm}
\sloppy

% definicje
\def\R3{\mathbb{R}^3}
\def\AGH{Akademia Górniczo-Hutnicza}

\newcounter{zad}
\newcommand{\zadanie}{\addtocounter{zad}{1} \medskip \noindent \textbf{\thezad. }}

\newcommand{\pytanie}[1]{\vspace{1mm}\zadanie~#1}

\newcommand{\odpowiedzi}[4]{
\vspace{2mm}
\begin{tabular}{|p{3mm}|p{14cm}|p{3mm}|}
\hline
a) & #1 & \\ \hline
b) & #2 & \\ \hline
c) & #3 & \\ \hline
d) & #4 & \\ \hline
\end{tabular}
\vspace{2mm}
}


\begin{document}


\section{Polecenie \emph{def}}

Współrzędnymi jednorodnymi (homogenicznymi) punktu skończonego $(x,y,z)$ w przestrzeni $\R3$ nazywamy dowolne cztery liczby (x',y',z',w) takie, że:
$$x = \frac{x'}{w},\qquad y = \frac{y'}{w},\qquad z = \frac{z'}{w}.$$

\AGH~im. Stanisława Staszica, Al. Mickiewicza 30, 30-059 Kraków


%%%%%%%%%%%%%%%%%%%%%%%%%%%%%%%%%%%%%%%%%%%%%%%%%%%%%%%%%%%%%%%%%%%%%%%%%%%%

\section{Polecenia \emph{newcounter} i \emph{newcommand}}

\zadanie (2 pkt.) Napisz deklaracje: 12-elementowej tablicy znaków (zmienna $a$) oraz funkcji $f$, która przyjmuje 2 argumenty $x$ i $y$ będące wskaźnikami do typu zmiennoprzecinkowego o podwójnej precyzji i która nie zwraca żadnej wartości.

\vspace{1mm} 

\dotfill \par \dotfill  

\vspace{1.5mm} 

\zadanie (2 pkt.) 
Zadeklaruj wskaźnik $p$ do typu zmiennoprzecinkowego pojedynczej precyzji, a następnie utwórz dynamicznie 50-elementową tablicę liczb zmiennoprzecinkowych, na którą wskazuje ten wskaźnik. Dopisz pętlę \emph{for} wypełniającą tablicę wartością 3.1416. Zadeklaruj wszystkie niezbędne zmienne.

\vspace{1mm}

\dotfill \par \dotfill \par \dotfill \par \dotfill \par \dotfill 

\vspace{1.5mm} 

\zadanie (2 pkt.) Napisz fragment kodu, w którym z pliku \emph{dane.txt} odczytywane są wartości dla dwóch zmiennych całkowitych $a$ i $b$. Zadeklaruj wszystkie niezbędne zmienne.

\vspace{1mm}

\dotfill \par \dotfill \par \dotfill 










\pytanie{Zaznacz te z poniższych odpowiedzi, które określają prawdziwe własności funkcji inline w języku C++.}

\odpowiedzi
{Funkcja inline jest to każda krótka funkcja, która mieści się w jednym wierszu (linii) kodu.}  
{Funkcja inline może być metodą zaimplementowaną bezpośrednio w definicji klasy.}  
{Funkcja inline jest to funkcja, której wywołanie jest zastępowane przez kompilator bezpośrednim wstawieniem jej instrukcji do kodu wynikowego.}  
{Funkcja inline może być funkcją wirtualną.}  





\end{document}

