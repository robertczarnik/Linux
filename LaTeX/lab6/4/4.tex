\documentclass{beamer}
\usepackage{txfonts} %for :=
\usepackage[utf8]{inputenc}
\usepackage[T1]{fontenc}
\usepackage[english]{babel}
\usepackage{graphicx}
\usepackage{times}
\usepackage{algorithm}
\usepackage{algpseudocode}
\usepackage{amsmath}
\usepackage{amssymb}
\usepackage{tikz}
\usetikzlibrary{calc,through,backgrounds,positioning,fit}
\usetikzlibrary{shapes,arrows,shadows,calendar}

\linespread{0.5}
\usetheme{Frankfurt}

\title{Beamer}
\author{A. Iksinsk}
\institute{Wydzial WAlilB \\
Katedra Informatyki Stosowanej}
\date{2015}

 
%--------------------------------------------------
\begin{document}

\begin{frame}
\titlepage	
\end{frame}


\begin{frame}[fragile]   % fragile!!!, bo używamy verb
\frametitle{Algorytm}

\onslide<1->\State $\triangleright$ \verb!ASSIGN! 
\onslide<2->\State $\triangleright$ \verb!init(s)! = \verb!s0;! 
\onslide<3->\State $\triangleright$ \verb!next(s)! \coloneqq \verb!case!

\onslide<4->	\ForAll{$si \in s$}
\onslide<5->		\ForAll{$tk \in T$}
\onslide<6->		\State $\mathbi{V}_{ik} \gets \emptyset$
\onslide<7->			\ForAll{$sj \in s$}
\onslide<8->                        \If{$(M_i, S_i) \overset{tk}{\longrightarrow} (M_j, S_j)$}
\onslide<9->                        \State $\mathbi{V}_{ik} \leftarrow \mathbi{V}_{ik} \cup \{sj\}$
\onslide<10->                \EndIf
        \EndFor
                \State $\triangleright$ \verb!s! $=$ \verb!si! \verb! & action! $=$ \verb!tk!$\colon  \{\mathbi{V}_{ik}$ \verb!contents!\};
\onslide<11->        \EndFor
\EndFor
\State $\triangleright$ \verb!esac;!


\end{frame}
 
%--------------------------------------------------
 
\begin{frame}
\frametitle{Zadanie 5.1}
 
\begin{tikzpicture}[scale=1,inner sep=0.4mm]
\onslide<1-3>{
  \coordinate (O) at (0,0);
  \coordinate (A) at (-75:2cm);
  \coordinate (B) at (85:2cm);
  \node at (O) [circle,fill=black!80!white] {};
  \node at (O) [left=2pt] {$O$};
  \draw [thick] (O) circle (2cm);
}
\onslide<2-3>{

	\node at (A) [circle, fill=black]{};
	\node at (A) [label={[label distance=0.03cm]135:\mathbi{$A$}}];
	
}

\onslide<3>{
	\node at (B) [circle, fill=black]{};
	\node at (B) [label={[label distance=0.03cm]225:\mathbi{$B$}}];
}
\onslide<4->{
	\fill [green!50, draw=grey] (O) -- (B) arc (85:-75:2) -- cycle;
	\draw [thick] (O) circle (2cm);
 	\node at (O) [circle,fill=black!80!white] {};
  	\node at (O) [left=2pt] {$O$};
	\node at (B) [circle, fill=black]{};
	\node at (B) [label={[label distance=0.03cm]225:\mathbi{$B$}}];
	\node at (A) [circle, fill=black]{};
	\node at (A) [label={[label distance=0.03cm]135:\mathbi{$A$}}];
	\node at (O) [right=5pt] {$\alpha = 160^\circ$};
}
\end{tikzpicture}
 
\end{frame}
 
%--------------------------------------------------
 
\begin{frame}
\frametitle{Kalendarz}
 
\begin{columns}
\column{0.45\textwidth}
\begin{block}{Styczeń 2015}
\begin{tikzpicture}
\calendar[dates=2015-01-01 to 2015-01-31,week list,day xshift=3.5ex]
	if (Saturday)[orange]
	if (Sunday)[red];
\end{tikzpicture}
\end{block}
\column{0.5\textwidth}

\only<2>{
  \begin{block}{Luty 2015}
  \tikz \calendar[dates=2015-02-01 to 2015-02-28,week list,day xshift=3.5ex]
	if (Saturday)[orange]
	if (Sunday)[red];
  \end{block}
}
\only<3>{
  \begin{block}{Marzec 2015}
  \tikz \calendar[dates=2015-03-01 to 2015-03-31,week list,day xshift=3.5ex]
	if (Saturday)[orange]
	if (Sunday)[red];
  \end{block}
}
\only<4>{
  \begin{block}{Kwiecien 2015}
  \tikz \calendar[dates=2015-04-01 to 2015-04-30,week list,day xshift=3.5ex]
	if (Saturday)[orange]
	if (equals=2015-04-06)[red]
	if (Sunday)[red];
  \end{block}
}
\end{columns}
\end{frame}
 
\end{document}